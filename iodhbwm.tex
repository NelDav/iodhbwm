\RequirePackage[utf8]{inputenc}
\documentclass[babel=ngerman,highlight=false]{skdoc}
\usepackage[T1]{fontenc}
\usepackage{lmodern}
\usepackage{enumitem}

\package[ctan=iodhbwm, vcs=https://github.com/faltfe/iodhbwm]{iodhbwm}
\version{0.1-alpha}

\title{iodhbwm bundle}
\author{Felix Faltin}
\email{ffaltin91@gmail.com}
\repository{https://github.com/faltfe/iodhbwm}

\usepackage{geometry}
\geometry{left=6cm, right=1.5cm}
\usepackage{blindtext}

%\KOMAoptions{parskip=half-}

\begin{document}
    \maketitle

    \begin{abstract}
        Bei dem Bundle \pkg{iodbwm} handelt es sich um eine \textbf{i}n\textbf{o}ffizielle Vorlage der \textbf{DHBW} \textbf{M}annheim zum Schreiben von Studien-, Praxis- und Bachelorarbeiten. Das Bundle stellt eine Klasse \pkg{iodbwm} und ein Paket \pkg{iodbwm-templates} bereit.

        Die vorgenommenen Einstellungen richtigen sie im Wesentlichen nach den Richtlinien der DHBW Mannheim zur Erstellung wissenschaftlicher Arbeiten.
    \end{abstract}

    \LongWarning{Das Bundle befindet sich derzeit noch in einer \textcolor{intlink}{Alpha}-Version. Änderungen sind jederzeit möglich.}

    \tableofcontents

    \section{Einleitung}
        Die Entwicklung des Bundle geschah ursprünglich aus persönlichen Gründen, denn mit jeder neuen Arbeit musste ich stets die gesamte Präamble meiner letzten Arbeit kopieren und gegebenenfalls Änderungen vornehmen. Außerdem war ich es leid, mir von Kommilitonen immer die gesamte Vorlage schicken lassen zu müssen, um dann doch festzustellen, dass die Dokumente doch nicht gleich aussehen.

        Deshalb kam ich zu dem Entschluss eine einfache Klasse zu entwickeln, welches das grundlegende Design entsprechend der Richtlinien der DHBW umsetzt. Zusätzlich dazu habe ich ein kleines Paket geschrieben, welches häufige Befehle definiert. Es wird empfohlen, dass das Paket in Verbindung mit der Klasse verwendet wird. Eine Voraussetzung ist es jedoch nicht.

    \section{Die Klasse iodbwm}\label{cls:iodhbwm}
        Die Angabe der Optionen erfolgt über das optionale Argument von \cs{documentclass}.
        Dabei wird auf das \meta{key}=\meta{value} System von \pkg{pgfopts} zurückgegriffen.
        
        \subsection{Optionen}
        
            \Option{load-preamble}\WithValues{true, false}\AndDefault{true}
            Bei Angabe der Option \opt{load-preamble} werden eine Reihe von zusätzlichen Paketen geladen und teilweise vorkonfiguriert. Nachfolgend erfolgt eine Auflistung der geladenen Pakete:
            \begin{description}[noitemsep]
                \item [\pkg{lmodern}]
                \item [\pkg{microtype}]
                \item [\pkg{srchack}]
                \item [\pkg{babel}]
                \item [\pkg{setspace}]
                \item [\pkg{scrlayer-srcpage}] Zusätzlich werden grundlegende Konfiguration zur Darstellung der Kopf- und Fußzeilen vorgenommen.
                \item [\pkg{geometry}] Die Seitenränder werden entsprechend der Richtlinien der DHBW eingestellt.
                \item [\pkg{siunitx}]
                \item [\pkg{mathtools}]
                \item [\pkg{graphicx}]
                \item [\pkg{tcolobox}] - Dieses Paket lädt implizit \pkg{tikz} und \pkg{xcolor}. Dem Paket \pkg{xcolor} werden die Optionen \opt{table} und \opt{dvipsnames} übergeben.
                \item [\pkg{tabularx}]
                \item [\pkg{booktabs}]
                \item [\pkg{multirow}]
            \end{description}

            \Option{load-dhbw-templates}\WithValues{true, false}\AndDefault{false}
            Bei Angabe der Option wird das Paket \pkg{iodhbwm-templates} geladen. Die dadurch bereitgestellten zusätzlichen Funktionen werden im Abschnitt~\ref{pkg:iodhbwm-templates} beschrieben.\bigskip

            \Option{add-bibliography}\WithValues{true, false}\AndDefault{false}
            \blindtext\bigskip

            \Option{bib-file}\WithValues{\meta{filename}}
            \blindtext\bigskip

            \Option{debug}
            Bei Angabe der Option werden die Pakete \pkg{blindtext} und \pkg{lipsum} geladen.
            
        \subsection{Allgemeine Makros}
            Derzeit stellt die Klasse keine Makros zur Verfügung.
            
        \subsection{Hintergrund Informationen}
            Die Klasse basiert auf der KOMA-Script Klasse \pkg{scrartcl}.

    \section{Das Paket iodbwm-templates}\label{pkg:iodhbwm-templates}
        \subsection{Optionen}
            Das Paket stellt das Makro \Macro{dhbwsetup}{\meta{key}=\meta{value}} bereit, über welches alle Einstellungen (Optionen) angepasst werden können. Hierfür sind eine Reihe von \meta{key} Variablen vordefiniert.\bigskip
            
            \Option{titlepage}\WithValues{\meta{filename}}\AndDefault{dhbw-titlepage.def}\bigskip
            
            \Option{declaration}\WithValues{\meta{filename}}\AndDefault{dhbw-declaration.def}\bigskip
            
            \Option{thesis type}\WithValues{SA, BA, PA}\bigskip
            
            \Option{thesis title}\WithValues{}\bigskip
            
            \Option{thesis second title}\WithValues{}\bigskip
            
            \Option{author}\WithValues{}\bigskip
            
            \Option{date}\WithValues{date}\AndDefault{\today}\bigskip
            
            \Option{location}\WithValues{}\bigskip
            
            \Option{institute}\WithValues{}\bigskip
            
            \Option{institute section}\WithValues{}\bigskip
            
            \Option{institute logo}\WithValues{}\bigskip
            
            \Option{student id}\WithValues{}\bigskip
            
            \Option{course}\WithValues{}\bigskip
            
            \Option{supervisor}\WithValues{}\bigskip
            
            \Option{processing period}\WithValues{}\bigskip
            
        \subsection{Allgemeine Makros}
            \DescribeMacro\dhbwsetup{\meta{key}=\meta{value}}
            
            \DescribeMacro\dhbwtitlepage{<filename>}
            
            \DescribeMacro\dhbwdeclaration
            
            \DescribeMacro\getAuthor
            
            \DescribeMacro\getDate
            
            \DescribeMacro\getThesisTitle
            
            \DescribeMacro\getThesisSecondTitle
            
            \DescribeMacro\getLocation
            
            \DescribeMacro\getSupervisor
            
            \DescribeMacro\getCourse
            
            \DescribeMacro\getStudentId
            
            \DescribeMacro\getInstitute
            
            \DescribeMacro\getInstituteSection
            
            \DescribeMacro\getProcessingPeriod
            
    \section{Beispiele und Anwendungen}
        \subsection{Eigene Titelseite definieren}
            \blindtext
            
        \section{Eigene Erklärung definieren}
            \blindtext
            
        \section{Abstract hinzufügen}
            \blindtext
            
    \section{Installation}
        \blindtext

    \section{Bekannte Probleme}
        \blindtext

    \PrintChanges
    \PrintIndex

\end{document}