\RequirePackage[utf8]{inputenc}
\documentclass[babel=ngerman,highlight=false]{skdoc}
\usepackage[T1]{fontenc}
\usepackage{lmodern}
\usepackage{enumitem}

\package[ctan=iodhbwm, vcs=https://github.com/faltfe/iodhbwm]{iodhbwm}
\version{0.2-alpha}

\title{iodhbwm bundle}
\author{Felix Faltin}
\email{ffaltin91@gmail.com}
\repository{https://github.com/faltfe/iodhbwm}

\usepackage{geometry}
\geometry{left=6cm, right=1.5cm}
\usepackage{blindtext}

%\KOMAoptions{parskip=half-}
\let\oldblindtext\blindtext
\renewcommand{\blindtext}{{\color{gray}\oldblindtext}}

\begin{document}
    \maketitle

    \begin{abstract}
        Bei dem Bundle \pkg{iodhbwm} handelt es sich um eine \textbf{i}n\textbf{o}ffizielle Vorlage der \textbf{DHBW} \textbf{M}annheim zum Schreiben von Studien-, Praxis- und Bachelorarbeiten. Das Bundle stellt eine Klasse \pkg{iodhbwm} und ein Paket \pkg{iodhbwm-templates} bereit.

        Die vorgenommenen Einstellungen richtigen sie im Wesentlichen nach den Richtlinien der DHBW Mannheim zur Erstellung wissenschaftlicher Arbeiten.
    \end{abstract}

    \LongWarning{Das Bundle befindet sich derzeit noch in einer \textcolor{intlink}{Alpha}-Version. Änderungen sind jederzeit möglich.}

    \tableofcontents

    \section{Einleitung}
        Die Entwicklung des Bundle geschah ursprünglich aus persönlichen Gründen, denn mit jeder neuen Arbeit musste ich stets die gesamte Präamble meiner letzten Arbeit kopieren und gegebenenfalls Änderungen vornehmen. Außerdem war ich es leid, mir von Kommilitonen immer die gesamte Vorlage schicken lassen zu müssen, um dann doch festzustellen, dass die Dokumente doch nicht gleich aussehen.

        Deshalb kam ich zu dem Entschluss eine einfache Klasse zu entwickeln, welches das grundlegende Design entsprechend der Richtlinien der DHBW umsetzt. Zusätzlich dazu habe ich ein kleines Paket geschrieben, welches häufige Befehle definiert. Es wird empfohlen, dass das Paket in Verbindung mit der Klasse verwendet wird. Eine Voraussetzung ist es jedoch nicht.

    \section{Die Klasse iodhbwm}\label{cls:iodhbwm}
        Die Angabe der Optionen erfolgt über das optionale Argument von \cs{documentclass}.
        Dabei wird auf das \meta{key}=\meta{value} System von \pkg{pgfopts} zurückgegriffen.
        
        \subsection{Optionen}
        
            \Option{load-preamble}\WithValues{true, false}\AndDefault{true}
            Bei Angabe der Option \opt{load-preamble} werden eine Reihe von zusätzlichen Paketen geladen und teilweise vorkonfiguriert. Nachfolgend erfolgt eine Auflistung der geladenen Pakete:
            \begin{description}[itemsep=1pt, style=multiline, leftmargin=3cm]
                \item [\pkg{lmodern}] Verwendung von Latin Modern anstatt Computer Modern
                \item [\pkg{microtype}] Verbesserungen des Schriftsatzes durch Änderungen der Abstände zwischen einzelnen Buchstaben und Wörtern
                \item [\pkg{babel}] Ermöglicht Sprachumschaltungen. Die Sprache kann als Option bei \verb|documenclass[language]{iodhbwm}| mitgeben werden.
                \item [\pkg{setspace}] Umschaltung zwischen einzeilig und anderthalbzeilig
                \item [\pkg{scrlayer-srcpage}] Zusätzlich werden grundlegende Konfiguration zur Darstellung der Kopf- und Fußzeilen vorgenommen.
                \item [\pkg{geometry}] Die Seitenränder werden entsprechend der Richtlinien der DHBW voreingestellt.
                \item [\pkg{siunitx}] Paket zum Schreiben von mathematischen Einheiten unter Beachtung der korrekten Schreibweise.
                \item [\pkg{mathtools}] Erweiterung des Standard zur Darstellung von mathematischen Ausdrücken
                \item [\pkg{graphicx}] Möglichkeit zur Einbindung von Bildern.
                \item [\pkg{tcolobox}] Dieses Paket lädt implizit \pkg{tikz} und \pkg{xcolor}. Dem Paket \pkg{xcolor} werden die Optionen \opt{table} und \opt{dvipsnames} übergeben.
                \item [\pkg{tabularx}] Erweiterung der Tabellenumgebung
                \item [\pkg{booktabs}] Möglichkeit zur Darstellung horizontaler Linien in Tabellen zur besseren Gestaltung
                \item [\pkg{multirow}] Paket zur vertikalen Verbinden von einzelnen Zellen in einer Tabelle
            \end{description}

            \Option{load-dhbw-templates}\WithValues{true, false}\AndDefault{false}
            Bei Angabe der Option wird das Paket \pkg{iodhbwm-templates} geladen. Die dadurch bereitgestellten zusätzlichen Funktionen werden im Abschnitt~\ref{pkg:iodhbwm-templates} beschrieben.\medskip

            \Option{add-bibliography}\WithValues{true, false}\AndDefault{false}
            Bei Aktivierung der Option wird versucht, ein Literaturverzeichnis zu erstellen. Dabei wenn die Option \opt{bib-file} nicht gesetzt ist, wird automatisch nach der Datei \file{dhbw-source.bib} gesucht.
            
            Das Literaturverzeichnis wird mittel \pkg{biblatex} und biber erstellt. Es ist darauf zu achten, dass die Einstellungen in der IDE gegebenenfalls anzupassen sind!\medskip

            \Option{bib-file}\WithValues{\meta{filename}}
            Der Option kann als \meta{key} eine Datei mitgegeben werden, welche die Einträge für das Inhaltsverzeichnis beinhalten. Es ist darauf zu achten, dass die Datei einschließlich Dateiendung übergeben wird.
            \begin{verbatim}
\documentclass[%
    add-bibliography = true,
    bib-file = my-source.bib
]{iodhbwm}
            \end{verbatim}
Diese Option ist nur in Verbindung mit \opt{add-bibliography} wirksam.\medskip

            \Option{debug}\WithValues{true, false}\AndDefault{false}
            Bei Angabe der Option werden die Pakete \pkg{blindtext} und \pkg{lipsum} geladen.
            
        \subsection{Allgemeine Makros}
            Derzeit stellt die Klasse keine Makros zur Verfügung.
            
        \subsection{Hintergrund Informationen}
            Die Klasse basiert auf der KOMA-Script Klasse \pkg{scrreprt}. Eine Änderung der Klasse ist in der derzeitigen Version \theversion{} nicht vorgesehen.

    \section{Das Paket iodhbwm-templates}\label{pkg:iodhbwm-templates}
        \subsection{Optionen}\label{pkg:options}
            Das Paket stellt das Makro \Macro\dhbwsetup{\meta{key}=\meta{value}} bereit, über welches alle Einstellungen (Optionen) angepasst werden können. Hierfür sind eine Reihe von \meta{key} Variablen vordefiniert.\medskip
            
            \Option{titlepage}\WithValues{\meta{filename}}\AndDefault{dhbw-titlepage.def}
            Mit der Option kann eine eigene Titelseite übergeben werden. Die Option \opt{thesis type} wird dabei ignoriert. Falls die angegeben Datei nicht gefunden wird, wird auf die Standardtitleseite zurückgegriffen.\medskip
            
            \Option{declaration}\WithValues{\meta{filename}}\AndDefault{dhbw-declaration.def}
            Mit der Option kann eine eigene Eigenständigkeitserklärung übergeben werden. In der derzeitigen Version wird nur eine deutsche Variante bereitgestellt.\medskip
            
            \Option{thesis type}\WithValues{SA, BA, PA}
            Die Option gibt die Art der Arbeit an. Die Abkürzungen sind wie folgt zu verstehen:
            \begin{description}[noitemsep,style=multiline,leftmargin=1cm]
                \item[SA] Studienarbeit
                \item[BA] Bachelorarbeit
                \item[PA] Praxisarbeit
            \end{description}
            Die Angabe des Typ der Arbeit bestimmt die Gestaltung der Titelseite.\medskip
            
            \Option{thesis title}\WithValues{\meta{value}}
            Die Option ermöglicht die Angabe des Titel (Thema) der Arbeit.\medskip
            
            \Option{thesis second title}\WithValues{\meta{value}}
            Im Fall einer Praxisarbeit \opt{thesis type} = \meta{PA} kann es vorkommen, dass zwei unterschiedliche Themen in einer Arbeit vorkommen. Das zweite Thema kann über diese Option definiert werden.\medskip
            
            \Option{author}\WithValues{\meta{value}}
            Mit der Option wird der Autor der Arbeit angegeben. Der Autor wird auf der Titelseite und im der Eigenständigkeitserklärung verwendet.\medskip
            
            \Option{date}\WithValues{\meta{value}}\AndDefault{\cs{today}}
            Mit der Option wird das Datum angegeben.\medskip
            
            \Option{location}\WithValues{\meta{value}}
            Mit Setzen der Option wird der Ort angegeben, an welchem die Arbeit erstellt wurde.\medskip
            
            \Option{institute}\WithValues{\meta{value}}
            Mit Angabe der Option wird der Firmenname angeben.\medskip
            
            \Option{institute section}\WithValues{\meta{value}}
            Eine weitere Spezialisierung des Firmennamen kann durch Angabe des Abteilung beschrieben werden. Die Abteilung kann mithilfe dieser Option angegeben werden.\medskip
            
            \Option{institute logo}\WithValues{\meta{filename}}
            Ein Firmenlogo kann dieser Option übergeben werden. Dieses wird automatisch auf den voreingestellten Titelseiten verwendet. Der \meta{filename} sollte ohne Dateiendung angegeben werden.\medskip
            
            \Option{student id}\WithValues{\meta{value}}
            Mit der Option wird die Matrikelnummer des Studenten angegeben.\medskip
            
            \Option{course}\WithValues{\meta{value}}
            Mit der Option wird die Kurskennung angegeben.\medskip
            
            \Option{supervisor}\WithValues{\meta{value}}
            Mit der Option wird der Betreuer der Arbeit angegeben.\medskip
            
            \Option{processing period}\WithValues{\meta{value}}
            Mit der Option wird der Zeitraum der Arbeit angegeben. Bei Arbeiten über zwei Semester kann die Angabe beispielsweise wie folgt erfolgen:
            \begin{verbatim}
\dhbwsetup{
    processing period = {01.01. - 31.03.17, 25.05. - 31.09.17}
}
            \end{verbatim}
            
        \subsection{Allgemeine Makros}
            \DescribeMacro\dhbwsetup{\meta{key}=\meta{value}} Das Makro ermöglicht die Angabe aller hier aufgelisteten Optionen einzustellen. Dabei werden die Option als \meta{key} angegeben und der einzustellende Wert als \meta{value}.
            
            \DescribeMacro\dhbwtitlepage Das Makro erstellt eine Titelseite. Dabei wird bei den vordefinierten Titelseiten (s.~\opt{thesis type}) auf die \textbf{zuvor} gesetzt Optionen zurück gegriffen. Eine eigene Definition einer Titelseite kann durch die Option \opt{titlepage} angegeben werden.
            
            \DescribeMacro\dhbwdeclaration Für das Setzen einer allgemeinen vordefinierten Selbstständigkeitserklärung (Eigenerklärung) ist das Makro zu verwenden. Eine eigene Definition kann mittels der Option \opt{declaration} übergeben werden.
            
            \DescribeMacro\getAuthor Abfrage des Autor, welcher durch \opt{author} übergeben wurde.
            
            \DescribeMacro\getDate Abfrage des Datum, welches durch \opt{date} übergeben wurde. Falls kein Datum angegeben wurde, wird \verb|\today| als Standard verwendet.
            
            \DescribeMacro\getThesisTitle Abfrage des Titel der Arbeit, welcher durch \opt{thesis title} übergeben wurde.
            
            \DescribeMacro\getThesisSecondTitle Abfrage des zweiten Titels, welcher durch \opt{thesis second title} übergeben wurde.
            
            \DescribeMacro\getLocation Abfrage des Orts, welcher durch \opt{location} übergeben wurde.
            
            \DescribeMacro\getSupervisor Abfrage des Betreuer, welcher durch \opt{supervisor} übergeben wurde.
            
            \DescribeMacro\getCourse Abfrage des Kurses, welcher durch \opt{course} übergeben wurde.
            
            \DescribeMacro\getStudentId Abfrage der Matrikelnummer, welche durch \opt{student id} übergeben wurde.
            
            \DescribeMacro\getInstitute Abfrage des Firmenname, welcher durch \opt{institute} übergeben wurde.
            
            \DescribeMacro\getInstituteSection Abfrage der Abteilung, welche durch \opt{institute section} übergeben wurde.
            
            \DescribeMacro\getProcessingPeriod Abfrage des Bearbeitungszeitraum, welcher durch \opt{author} übergeben wurde.
            
    \section{Beispiele und Anwendungen}
        \subsection{Eigene Titelseite definieren}
            Es kann vorkommen, dass man die Klasse verwenden möchte, jedoch die vordefinierten Titelseiten einem nicht gefallen oder modifizieren möchte. Hierzu stehen einem zwei Varianten zur Verfügung.
            
            \minisec{Titelseite mit \cs{maketitle}}
            Dabei wird auf das herkömmliche Makro \verb|\maketitle| zurückgegriffen. Allerdings ist es dann notwendig, dass die Attribute selbstständig gesetzt werden.
            \begin{verbatim}
\title{Die DHBW ist toll}
\author{Max Mustermann}
\date{\today}
...
\maketitle
            \end{verbatim}
            
            \minisec{Titelseite mit der Umgebung \env{titlepage}}
            Diese Variante bietet eine größere gestalterische Freiheit. Das Grundgerüst kann den beiliegenden Templates entnommen werden. Anschließend kann dann über die Option \opt{titlepage} = \meta{filename} die eigene Titelseite angegeben werden. Die Dateiendung kann bei Angabe des \meta{filename} weggelassen werden.
            
        \section{Eigene Erklärung definieren}
            Eine eigene (Eigenständigkeits-) Erklärung kann über die Option \opt{declaration} = \meta{filename} übergeben werden. Auf die Angabe der Dateiendung kann verzichtet werden.
            
%        \section{Abstract hinzufügen}
%            \blindtext
            
    \section{Installation}
        \subsection{Lokale Installation}
            Eine eigene Installation des Paket kann in einem lokalen texmf Ordner (lokales Repository) erfolgen. Das Bundle kann manuell aus dem Git-Repository heruntergeladen werden.
            
            \subsubsection{MiKTeX}
                \begin{enumerate}
                    \item Lokales Repository anlegen, welches der \href{http://tug.ctan.org/tds/tds.html}{Verzeichnisstruktur für \LaTeX Dateien} entspricht. Die Verzeichnisstruktur könnte wie folgt aussehen:\par \verb|C:\Users\<username>\localtexmf\tex\latex\iodhbwm|
                    \item MiKTeX Settings öffnen
                    \item Unter dem Reiter ,,Roots'' das Verzeichnis hinzufügen\par \verb|C:\Users\<username>\localtexmf|
                    \item Anschließend unter ,,General'' auf den Button Refresh FNDB klicken
                \end{enumerate}
            
                Der letzte Schritt muss immer wieder ausgeführt werden, wenn ein neues Release heruntergeladen wurde.
                
                Eine ausführliche Beschreibung befindet sich auf \url{https://tex.stackexchange.com/a/69484/142408}.
            
            \subsubsection{TeXlive}
                \Warning{Die Installationsanweisung für TeXlive wurde nicht getestet!}
                
                \begin{enumerate}
                    \item \verb|kpsewhich -var-value TEXMFHOME| zum Auffinden des Ordner
                    \item ~/texmf/tex/latex/ (ggf. neu anlegen)
                    \item  Ordner iodhbwm nach kopieren ~/texmf/tex/latex/ (Ordner muss zwingend iodhbwm heißen)
                    \item texhash ~/texmf ausführen
                \end{enumerate}
            
                Eine ausführliche Beschreibung befindet sich auf \url{https://tex.stackexchange.com/a/73017/142408}.
                
        \subsection{CTAN}
            Das Bundle wird ebenfalls über CTAN (mit Release der Version v0.1) zur Verfügung gestellt und kann deshalb über die offiziellen Paketquellen heruntergeladen und installiert werden. Diese Variante ist zu bevorzugen.

%    \section{Bekannte Probleme}
%        \blindtext

    \PrintChanges
    \PrintIndex

\end{document}